\documentclass[a4paper, 11 pt]{article}
\usepackage{datetime}
\usepackage{lipsum}
\setlength{\columnsep}{25 pt}
\usepackage{tabularx,booktabs,caption}
\usepackage{multicol,tabularx,capt-of}
\usepackage{multirow}
\usepackage{subcaption}
\usepackage[utf8]{inputenc}
\usepackage{hyperref}
\usepackage{blkarray}
\usepackage[margin=1.2in]{geometry}
\usepackage{amsthm, amsmath, amssymb, amsfonts, commath, amsthm}
\usepackage{placeins}
\usepackage{graphicx}
\usepackage{mathrsfs}
\usepackage{listings}
\usepackage{lastpage}
\usepackage{enumerate}
\usepackage{subcaption}
\usepackage[danish]{babel}
%\renewcommand{\qedsymbol}{\textit{Q.E.D}}
\newtheorem{theorem}{Theorem}[section]
\addto\captionsdanish{\renewcommand\proofname{Proof}}
\newtheorem{lemma}{Lemma}[section]
\newtheorem{corollary}{corollary}[section]
\usepackage{fancyhdr}
\usepackage{titlesec}
\newcommand{\code}[1]{\texttt{#1}}
\lhead{Anders Gantzhorn - tzk942}
\chead{}
\rhead{09-11-2022}
\pagestyle{fancy}
\setlength{\headheight}{15pt}
\cfoot{\thepage\ af \pageref{LastPage}}
\setcounter{section}{0}
\title{Stochsatic gradient descent for logistic regression smoothing}
\author{Anders G. Kristensen}
\date{09-11-2022}
\begin{document}
\maketitle
\section{The logistic regression model}
\noindent We consider the logistic regression $y_i\in\{0,1\}, x_i\in\mathbb{R}$ s.t. $p_i(\beta) = P(Y_i = 1| X_i = x_i)$ and
\[
    \log\left(\frac{p_i(\beta)}{1-p_i(\beta)}\right) = f(x_i|\beta) = \left(\varphi_1(x_i), \dots, \varphi_p(x_i)\right)^\top\beta    
\] 
For some $\beta\in\mathbb{R}^p$ and fixed basis functions $\varphi_1(x_i), \dots, \varphi_p(x_i) : \mathbb{R} \to \mathbb{R}$. We aim to minimize the penalized log-likelihood.
\[
    H(\beta) = -\frac{1}{N}\sum_{i = 1}^N \left(y_i\log(p_i(\beta))+(1-y_i)\log(1-p_i(\beta))\right) + \lambda||f''_\beta||_2^2 
\]
over $\beta\in\mathbb{R^p}$.
\section{Horse data}
The dataset contains a couple of missing values we need to handle appropriately.
\begin{table}[ht]
    \centering
    \begin{tabular}{rllr}
      \hline
     & Dead & Missing & Count \\ 
      \hline
    & FALSE & FALSE & 426 \\ 
    & FALSE & TRUE &  17 \\ 
    & TRUE & FALSE & 106 \\ 
    & TRUE & TRUE &   2 \\ 
       \hline
    \end{tabular}
    \end{table}
\end{document}